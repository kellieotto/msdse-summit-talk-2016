\documentclass{beamer}

\usepackage{beamerthemesplit}
\usepackage{graphicx}
\usepackage{color, natbib, hyperref}
\usepackage{bibentry}
\nobibliography*

% define colors
\definecolor{jblue}  {RGB}{20,50,100}
\definecolor{ngreen} {RGB}{98,158,31}

%theme

\usetheme{boxes} 
%\usecolortheme{seahorse} 
\setbeamertemplate{items}[default] 
%\setbeamercovered{transparent}
\setbeamertemplate{blocks}[rounded]
\setbeamertemplate{navigation symbols}{} 
% set the basic colors
\setbeamercolor{palette primary}   {fg=black,bg=white}
\setbeamercolor{palette secondary} {fg=black,bg=white}
\setbeamercolor{palette tertiary}  {bg=jblue,fg=white}
\setbeamercolor{palette quaternary}{fg=black,bg=white}
\setbeamercolor{structure}{fg=jblue}
\setbeamercolor{titlelike}         {bg=jblue,fg=white}
\setbeamercolor{frametitle}        {bg=jblue!10,fg=jblue}
\setbeamercolor{cboxb}{fg=black,bg=jblue}
\setbeamercolor{cboxr}{fg=black,bg=red}

% reduce space before/after equations
\expandafter\def\expandafter\normalsize\expandafter{%
    \normalsize
    \setlength\abovedisplayskip{1pt}
    \setlength\belowdisplayskip{1pt}
    \setlength\abovedisplayshortskip{1pt}
    \setlength\belowdisplayshortskip{1pt}
}

% set colors for itemize/enumerate
\setbeamercolor{item}{fg=ngreen}
\setbeamercolor{item projected}{fg=white,bg=ngreen}

% set colors for blocks
\setbeamercolor{block title}{fg=ngreen,bg=white}
\setbeamercolor{block body}{fg=black,bg=jblue!10}

% set colors for alerted blocks (blocks with frame)
\setbeamercolor{block alerted title}{fg=white,bg=jblue}
\setbeamercolor{block alerted body}{fg=black,bg=jblue!10}
\setbeamercolor{block alerted title}{fg=white,bg=dblue!70} % Colors of the highlighted block titles
\setbeamercolor{block alerted body}{fg=black,bg=dblue!10} % Colors of the body of highlighted blocks

% set the fonts
\usefonttheme{professionalfonts}

\setbeamerfont{section in head/foot}{series=\bfseries}
\setbeamerfont{block title}{series=\bfseries}
\setbeamerfont{block alerted title}{series=\bfseries}
\setbeamerfont{frametitle}{series=\bfseries}
\setbeamerfont{frametitle}{size=\Large}
\setbeamerfont{block body}{series=\mdseries}
\setbeamerfont{caption}{series=\mdseries}
\setbeamerfont{headline}{series=\mdseries}


% set some beamer theme options
\setbeamertemplate{title page}[default][colsep=-4bp,rounded=true]
\setbeamertemplate{sections/subsections in toc}[square]
\setbeamertemplate{items}[circle]
\setbeamertemplate{blocks}[width=0.0]
\beamertemplatenavigationsymbolsempty

% Making a DAG
\usepackage{tkz-graph}  
\usetikzlibrary{shapes.geometric}
\usetikzlibrary{positioning}
\tikzstyle{VertexStyle} = [shape            = rectangle,
                               minimum width    = 6ex,%
                               draw]
 \tikzstyle{EdgeStyle}   = [->,>=stealth']      

% Define block styles
\tikzstyle{f} = [rectangle, draw, fill=blue!20, 
    text width=3em, text badly centered, node distance=1.75cm]
\tikzstyle{message} = [rectangle, draw, fill=green!20, 
    text width=3em, text centered]
\tikzstyle{io} = [draw, circle,fill=red!20, node distance=2cm,
    minimum height=2em]
\tikzstyle{line} = [draw, -latex']

% Math macros
\newcommand{\cD}{{\mathcal D}}
\newcommand{\cF}{{\mathcal F}}
\newcommand{\todo}[1]{{\color{red}{TO DO: \sc #1}}}

\newcommand{\reals}{\mathbb{R}}
\newcommand{\integers}{\mathbb{Z}}
\newcommand{\naturals}{\mathbb{N}}
\newcommand{\rationals}{\mathbb{Q}}

\newcommand{\ind}[1]{1_{#1}} % Indicator function
\newcommand{\pr}{\mathbb{P}} % Generic probability
\newcommand{\ex}{\mathbb{E}} % Generic expectation
\newcommand{\var}{\textrm{Var}}
\newcommand{\cov}{\textrm{Cov}}

\newcommand{\normal}{N} % for normal distribution (can probably skip this)
\newcommand{\eps}{\varepsilon}
\newcommand\independent{\protect\mathpalette{\protect\independenT}{\perp}}
\def\independenT#1#2{\mathrel{\rlap{$#1#2$}\mkern2mu{#1#2}}}

\newcommand{\convd}{\stackrel{d}{\longrightarrow}} % convergence in distribution/law/measure
\newcommand{\convp}{\stackrel{P}{\longrightarrow}} % convergence in probability
\newcommand{\convas}{\stackrel{\textrm{a.s.}}{\longrightarrow}} % convergence almost surely

\newcommand{\eqd}{\stackrel{d}{=}} % equal in distribution/law/measure
\newcommand{\argmax}{\arg\!\max}
\newcommand{\argmin}{\arg\!\min}


\mode<presentation>

\title[Simple Random Sampling: Not So Simple]{Simple Random Sampling: Not So Simple \\ MSDSE 2016}
\author{Kellie Ottoboni \\ with Philip B.~Stark and Ron Rivest}
\institute[]{Department of Statistics, UC Berkeley\\Berkeley Institute for Data Science}
\date{October 24, 2016}

\begin{document}

\frame{\titlepage}




\section[Introduction]{Introduction}
\frame
{
  \frametitle{PRNGs}
  
If you use a computer to do random sampling (including permutation tests, bootstrapping, MCMC, basically any kind of simulation), you use prngs.
\begin{center}
\begin{itemize}
\item \textbf{Pseudorandom}: computationally indistinguishable from the uniform distribution
\item other desirable properties
\end{itemize}
\end{center}
}



\frame{
\frametitle{The good, the bad, and the ugly}
\begin{block}{\citet{knuth_art_1997}}
``Random numbers should not be generated with a method chosen at random.''
\end{block}

\vspace{15pt}
\textbf{RANDU}: the sequence $(x_n)$ given by

$$x_{n+1} = (65539 x_{n}) \mod 2^{31}.$$
\vspace{10pt}

\footnotesize{$0.003051898, 0.018310966, 0.082398718, 0.329593616, 0.235973230,...$}
\pause

\small
\begin{figure}[htbp]
\begin{center}
\includegraphics[width = 0.4\textwidth]{randu.png}
\end{center}
     Triples of RANDU lie on 15 planes in 3D space \small{(Wikipedia)}
\end{figure}
}

\frame{
\frametitle{Pigeons and Pigeonholes}

\begin{theorem}[Pigeonhole Principle]
If there are $n$ pigeonholes and $m>n$ pigeons, then there exists at least one pigeonhole containing more than one pigeon.
\end{theorem}
\begin{figure}[htbp]
\begin{center}
\includegraphics[width = .3\textwidth]{TooManyPigeons.jpg}
\end{center}
\tiny \href{https://commons.wikimedia.org/w/index.php?curid=4658682}{(Wikipedia)}
\end{figure}

\pause
\begin{corollary}[Too few pigeons]
If ${n \choose k}$ is greater than the size of a PRNG's state space, then the PRNG cannot possibly generate all samples of size $k$ from a population of $n$.
\end{corollary}
}


\frame{
\frametitle{Pigeons and Pigeonholes}

How much does it matter? 
\pause

\vspace{20pt}

Period of 32-bit linear congruential generators (e.g. RANDU): $2^{32} \approx 4 \times 10^9$ \\
Samples of size $10$ from $50$: ${50 \choose 10} \approx 10^{10}$ \\
\vspace{20pt}
\pause

Period of Mersenne Twister (standard PRNG in Statistics): $2^{32 \times 624} \approx 2 \times 10^{6010}$ \\
Permutations of $2084$ objects: $2084! \approx 3 \times 10^{6013}$


}


\frame{
\frametitle{A better alternative}

\textbf{Solution:} Find a class of PRNGs with infinite state space!


\begin{center}
\resizebox{10cm}{!}{    
\begin{tikzpicture}[node distance = 1cm, auto, scale = 0.5]
    % Place nodes
    \node [io] (IV) {IV};
    \node [f, right of=IV] (f1) {f};
    \node [f, right of=f1] (f2) {f};
%    \node [f, right of=f2, minimum width = 0cm, height = 0cm] (invisible) {};
    \node [f, right of=f2, node distance=3cm] (fn1) {f};
    \node [f, right of=fn1] (fend) {f};
%    \node [f, right of=invisible, node distance=3cm] (fend) {f};
    \node [f, right of=fend] (g) {g};
    \node [io, right of=g] (hx) {$h(x)$};
    \node [message, above of=f1] (m1) {$x_1$};
    \node [message, above of=f2] (m2) {$x_2$};
    \node [message, above of=fn1] (mn1) {$x_{n-1}$};
    \node [message, above of=fend] (mend) {$x_n$};
    \node [message, above right = of m1, above left = of mend, minimum width = 5cm] (message) {$x$};
    % Draw edges
    \path [line] (IV) -- (f1);
    \path [line] (f1) -- (f2);
    \path [line] (message) -- (m1);
    \path [line] (message) -- (m2);
    \path [line] (message) -- (mn1);
    \path [line] (message) -- (mend);
    \draw [-,dotted] (m2) -- (mn1);
    \path [line] (m1) -- (f1);
    \path [line] (m2) -- (f2);
     \path [line] (mn1) -- (fn1);
    \path [line] (mend) -- (fend);
    \path [line] (fend) -- (g);
    \path [line,dashed] (f2) -- (fn1);
    \path [line] (fn1) -- (fend);
    \path [line] (g) -- (hx);
\end{tikzpicture}
}
\end{center}


Cryptographic hash functions:
\begin{itemize}
\item computationally infeasible to invert
\item difficult to find two inputs that map to the same output
\item small input changes produce large, unpredictable changes to output
\item resulting bits are uniformly distributed
\end{itemize}

}

\frame{
\frametitle{Open questions}
\begin{itemize}
\itemsep10pt
\item Sampling algorithms: do some give samples with unequal probability?
\item For PRNGs with sufficiently large state space, do they produce all samples with equal probability? All permutations?
\item Are departures from uniformity large enough to bias statistics of interest?
\item Replace the default PRNGs in Python
\url{https://www.github.com/statlab/cryptorandom}
\end{itemize}

}

\begin{frame}
\frametitle{References}
\tiny
\bibliographystyle{plainnat}
\bibliography{refs}
\itemize
\end{frame}


\end{document}
